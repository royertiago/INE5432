\documentclass[utf8]{beamer}
\usetheme[compress]{Singapore}

\usepackage[brazil]{babel}

\begin{document}

\author{Tiago Royer}
\title{$k$-Nearest Neighbors}
\date{8 de junho de 2015}
\institute{UFSC}
\begin{frame}
    \titlepage
\end{frame}

\begin{frame}
    \frametitle{Síntese}
    \tableofcontents
\end{frame}

\section{Definição do predicado}

\begin{frame}
    \frametitle{Definição do predicado}

    Seja $k$ um inteiro positivo,\\
    $a$ um parâmetro,\\
    e $E$ um conjunto.

    \begin{equation*}
        \sigma_{k, a}(E)
    \end{equation*}

    são os $k$ elementos de $E$ mais próximos de $a$.
\end{frame}

\section{Algoritmos e estruturas de dados}

\begin{frame}
    \frametitle{Algoritmos e estruturas de dados}
\end{frame}

\section{Aplicações}
\begin{frame}
    \frametitle{Aplicações}
\end{frame}

\end{document}
